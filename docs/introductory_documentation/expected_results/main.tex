\section{Expected results}

\subsection{Dynamic}
We most likely will manage to implement dynamic algorithm both for GPU and CPU. The GPU implementation of dynamic algorithm will probably perform very well as long as memory will allow it to do so. The GPU implementation is our first modification to the well-known algorithm and we expect, that it will achieve much better performance results than its CPU version. We predict that the limits on memory will be reached for graphs with around 30 vertices. The exact value depends on union-find structure implementation.

\subsection{Branch and bound}
This algorithm has higher complexity than dynamic algorithm and most likely it will perform worse than dynamic algorithm, however, it does not have limits on memory as its memory complexity is polynomial. We may manage to provide CPU implementation, but because of its recursive nature it may cause a lot of trouble to accomplish GPU one. This algorithm was designed to address memory issues that the dynamic algorithm imposes.

\subsection{Hybrid}
It is expected, that this algorithm will perform much better, than its ancestor, branch and bound algorithm. Its improvement will be strictly determined by the number of iterations the dynamic algorithm was capable of carrying out. This algorithm is planned to be implemented on a GPU thus further reducing its running time.

\subsection{General thoughts}
Having presented our ideas for computing treedepth along with its decomposition, we acknowledge that they highly vary in degree of complexity. Some of them are very easy to be implemented both on GPU and CPU, and some seem to pose a challenge to be implemented on a GPU. Nevertheless we aim to accomplish all of the goals set, including the hardest of all: \textbf{hybrid approach implemented on a GPU}.
