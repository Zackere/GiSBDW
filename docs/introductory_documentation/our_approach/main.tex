\section{Our approach}
\subsection{Dynamic Algorithm}
\subsubsection{Binomial Encoding \cite{binomial_encoding}}
In this section we shall present a simple method to optimally encode $k$-element subsets of $\{0,...,n-1\}$ for $k\in\left[0;n\right]$ called binomial encoding:
\begin{lstlisting}[language=C++]
size_t Encode(Set s, size_t n, size_t k) {
	size_t ret = 0;
	while (k > 0) {
		--n;
		if (s.contains(n)) {
			ret += NChooseK(n, k);
			--k;
		}
	}
	return ret;
}

Set Decode(size_t code, size_t n, size_t k) {
  Set ret;
  while (k > 0) {
    --n;
    size_t nk = NChooseK(n, k);
    if (code >= nk) {
      ret.insert(n);
      code -= nk;
      --k;
    }
  }
  return ret;
}
\end{lstlisting}
This method of encoding is optimal because it forms a bijection: $\binom{\{0,...,n-1\}}{k}\rightarrow\left[0; \binom{n}{k}\right)$.\\
It is capable of running in $O(n)$ time, provided that Pascals' Triangle is precomputed.
\subsubsection{UnionFind data structure}
UnionFind data structure is crucial in this approach. It will allow us to compute treedepth efficiently. With each set it associates some value, which will be managed by UnionFind data structure internally. Our implementation shall satisfy following contract:
\begin{itemize}
	\item It can be constructed from integer $n$. As a result, we get a UnionFind data structure representing $n$ disjoint one-element subsets of $\{0,...,n-1\}$ and have associated value equal to 1. Each set has an id equal to the element it contains.
	\item \texttt{UnionFind Clone()} clones given object.
	\item \texttt{ValueType GetValue(SetIdType)} returns value associated with given set.
	\item \texttt{ValueType GetMaxValue()} returns maximum of all values associated with sets represented by the object.
	\item \texttt{SetIdType Find(ElemType)} returns id of a set in which given element is contained.
	\item \texttt{SetIdType Union(SetIdType s1, SetIdType s2)} sums two given sets. As a result of this operation:
	\begin{itemize}
		\item \texttt{s1} will not change its id.
		\item Every element of \texttt{s2} will be contained in \texttt{s1}.
		\item Value associated with \texttt{s1} is replaced with \\ the greatest of values: \texttt{GetValue(s1)}, \texttt{GetValue(s2) + 1}
	\end{itemize}
\end{itemize}
The data structure defined this way allows to compute treedepth very efficiently, as values associated with contained sets change in the same way as treedepth if we were to attach one tree to the root of another tree. Complexity of each operation is not much bigger than $O(1)$ (if not exactly $O(1)$) except for construction and copying which have $O(n)$ time complexity.
\subsubsection{Idea behind the algorithm}
Having described all necessary tools, we shall now proceed to describe dynamic approach to computing treedepth value as well as treedepth decomposition. As mentioned earlier, this method trades off space complexity of $O\left(2^{\left|G\right|}\cdot\left|G\right|\right)$ for time complexity of  $O^{*}\left(2^{\left|G\right|}\right)$.
The algorithm builds the treedepth decomposition in bottom-to-top manner. It generates an array of pairs \texttt{(UnionFind, size\_t)} from another array of the same type. We start off with an array of one element: unmodified UnionFind constructed from $\left|G\right|$ and index 0. Then, given a UnionFind object, for every inactive vertex $v$ in it, we shall generate a new one by:
\begin{itemize}
	\item activating $v$ in considered object
	\item performing \texttt{Union(Find($v$), Find($w$))} for every active neighbor $w$ of $v$ in $G$
\end{itemize}
This very action adds vertex $v$ to some treedepth decomposition of $G$ induced on set of active vertices.\\
At this point it would be appropriate to explain what an active vertices are and how we will handle them. It is a concept which tells us whether a vertex is in a solution represented by given UnionFind object. With an index in an array we associate a UnionFind object. This index shall be decoded into a subset of $\{0,...,n-1\}$ which will tell us what vertices are active in object in question (nevertheless nothing stops us from accessing other elements).\\
Having generated a new UnionFind we shall store it in output array under the index equal to encoding of a set of active vertices contained in generated UnionFind object. With it we store an index from which we received the original UnionFind object. It will be useful in treedepth decomposition reconstruction. If there is some UnionFind object in the designated space, we replace the object in it only if our UnionFind has lower max value.
\subsubsection{The Algorithm}
\begin{lstlisting}[language=C++]
void TDDynamicStep(Graph g, size_t step_number, (UnionFind, size_t)[] prev, (UnionFind, size_t)[] next) {
  for(size_t subset_index = 0; subset_index < prev.size(); ++subset_index) {
    // Get set of active vertices
    Set s = Decode(subset_index, V(g).size(), step_number);
    for(ElemType x : V(g)\s) {
      UnionFind uf_x = prev[subset_index].first.Clone();
      Set ids = {};
      // Find trees containing neighbors of v
      for(ElemType v : g.neigh(x))
        if(s.contains(v))
          ids.insert(uf_x.Find(v));
      SetIdType new_set_id = uf_x.Find(x);
      // Ensure that we are still in compliance with treedepth decomposition definition when adding x to active vertices
      for(SetIdType id : ids)
        new_set_id = uf_x.Union(new_set_id, id)
      s.insert(x);
      // Determine designated index of our new object
      size_t dest = Encode(s, V(g).size(), step_number + 1);
      // Check if we have improved current result for combination of active vertices represented by s
      if(next[dest] == null || next[dest].first.GetMaxValue() > uf_x.GetMaxValue())
        next[dest] = (uf_x, subset_index);
    }
  }
}

(UnionFind, size_t)[][] TDDynamicAlgorithm(Graph g) {
  // Starting point of the algorithm: object representing empty set of active vertices
  (UnionFind, size_t)[][] history = new (UnionFind, size_t)[V(g).size()][];
  history[0] = new (UnionFind, size_t)[1];
  history[0][0] = UnionFind(V(g).size());
  for(size_t i = 0; i < V(g).size(); ++i) {
    history[i + 1] = new (UnionFind, size_t)[NChooseK(V(g).size(), i + 1)];
    TDDynamicStep(g, i, history[i], history[i + 1]);
  }
  return history;
}
\end{lstlisting}
Having completed the dynamic algorithm, the object on the last page of history holds the treedepth of our graph. Based on the history we can reconstruct the treedepth decomposition in $O(|G|)$ time:
\begin{lstlisting}[language=C++]
(Graph, int) TDDecomposition(Graph g, (UnionFind, size_t)[][]history) {
  size_t current_entry = 0, current_page = V(g).size() - 1;
  int depth = history[current_page][current_entry].second;
  // This list will hold permutation of vertices that made up the object in history[current_page][current_entry]
  List indices = {current_entry};
  while(current_page > 0) {
    indices.push_front(history[current_page][current_entry].second);
    current_entry = history[current_page][current_entry].second;
    --current_page;
  }
  Graph tree(V(g).size());
  while(current_page < V(g).size()) {
    size_t current_entry = indices.pop_front();
    Set s = Decode(current_entry, V(g).size(), current_page);
    // Find vertex that had been added in considered transition
    ElemType v = Decode(indices.front(), V(g).size(), current_page)\s;
    UnionFind uf = history[current_page][current_entry].first;
    // Add edges required by the definition of treedepth decomposistion
    for(ElemType w : g.neigh(v))
      if(s.contains(w))
        tree.add_edge(v, uf.Find(w));
    ++current_page;
  }
  return tree, depth;
}
\end{lstlisting}
This algorithm finds what vertices have been added on consequent steps and joins appropriate trees under common parent vertex.
\subsubsection{Remarks}
As we can see, this algorithm provides an obvious improvement in time complexity of $O^{*}\left(2^{\left|G\right|}\right)$ compared to na\"ive approach with time complexity of $O\left(\left|G\right|!\right)$, but it requires additional $O\left(2^{\left|G\right|}\cdot\left|G\right|\right)$ space to run. It is particularly important for set encoding to be very efficient as it will be executed many times, therefore, presented method for such encoding might not make it into final implementation and be replaced with different method or be modified significantly. Space complexity of this algorithm makes it impractical to be run on large graphs. We address this issue in next sections of this document. It also has some useful properties which we will exploit, but this will be also discussed later.
\subsection{Branch and Bound algorithm}
In contrary to dynamic algorithm, this time we will build treedepth decomposition in top-to-bottom manner. To make this approach efficient, we shall incorporate branch and bound technique. Please note that this approach assumes connected input graph. This assumption is valid as for disconnected graphs we can run this algorithm for every connected component.
\subsubsection{Treedepth decomposition of incomplete elimination}
In this section we shall define notions that will be crucial in further sections of this document.
\begin{enumerate}
	\item Incomplete elimination $w$ is a word with distinct characters upon alphabet $V(G)$.
	\item Treedepth decomposition of incomplete elimination $w$ is a rooted tree $T_w$ such that:
	\begin{itemize}
		\item Treedepth decomposition of empty elimination is a single vertex represented by $G$.
		\item Treedeph decomposition of incomplete elimination $wv$, is built from $T_w$, in the following way:
		\begin{enumerate}
			\item Find leaf $H$ in $T_w$ such that $v\in H$.
			\item Add every connected component of $H-v$ as a child of $H$.
			\item Replace node $H$ with $v$.
		\end{enumerate}
	\end{itemize}
	\item Depth of a node $v$ in $T_w$ is defined as a length of a path between $T_w$ root and $v$ and is denoted as $depth_{T_w}\left(v\right)$.
\end{enumerate}
Those notions embrace the history of treedepth decomposition, i.e. how it used to be called elimination tree. The $T_w$ structure really shows how vertex elimination affects the graph while preserving information about the elimination and is capable of illustrating an elimination in progress. $T_w$ of complete elimination (when $\left|w\right|=\left|G\right|$), after removal of empty leaves, is a treedepth decomposition of $G$ (or maybe rather an elimination tree at this point).
\subsubsection{The algorithm}
As in every branch and bound algorithm, we shall start with some valid solution to our problem. This solution will be improved upon and be used to eliminate some unnecessary attempts. Our initial solution shall be $P_{\left|G\right|}$.\\\\
Let $L$ be a language of permutations of $V\left(G\right)$. We shall traverse prefix tree built upon $L$.
Let $w=\varepsilon$ (empty word). For each $H$ such that $H$ is a leaf of $T_w$, if $lbtd(H)+depth_{T_w}(H)$ is bigger than currently best known upper-bound of $td(G)$, then, if $w=\varepsilon$ terminate algorithm, otherwise remove last letter from $w$ and continue tree traversal. If $|w|=|G|$, then update current best known treedepth decomposition with treedepth decomposition defined by $w$ if necessary. If $|w|<|G|$, then, for each $v\notin w$, traverse subtree rooted in $wv$.\\\\
Note: $lbtd(H)$ denotes lower bound of $td\left(H\right)$ and for non-empty graphs is at least 1. When $td\left(H\right)$ is known, it can be used instead of $lbtd(H)$.\\
Note: Upon elemination, one can find the deepest leaf $H$ in $T_w$ with the most vertices. Then $\left|H\right| + depth_{T_w}(H)$ is an upper-bound of $td(G)$, possibly improving upon it.
\subsubsection{Remarks}
This algorithm has very high pessimistic time complexity of $O\left(\left|G\right|!\right)$, but has small (polynomial) space complexity. Knowledge about treedepth value of induced subgraphs of $G$ as well as fast discovery of low-treedepth solutions can greatly benefit performance of this algorithm.
\subsection{Modifications}
This section is devoted to acknowledging some interesting properties of just described algorithms. Since we aim to incorporate GPUs into our project it revolves around this idea.
\subsubsection{Dynamic algorithm}
The Dynamic algorithm is very easy to parallelize on level of one step. Each subset can be processed independently of the others. The only synchronization will be needed to update output array but this should impose no problem at all. Another interesting property of this algorithm is possibility of reconstructing treedepth decompositions in parallel. If we were to terminate this algorithm at, let's say 18-element subsets of 50-element set, we can, for each of these subsets, reconstruct the treedepth decomposition independently of others, provided that operation \texttt{SetId Find(ElemType)} will not change the UnionFind object i.e. will not perform path compression if told not to do so.
\subsubsection{Branch and Bound algorithm}
Obviously we will make an attempt to implement this algorithm on a GPU. It may require some modifications to it, but we will not be deterred. The most important modification will be combining this algorithm and the dynamic algorithm together, to create so called \emph{Hybrid Algorithm}. This modification provides treedepth values of induced subgraphs requested by branch and bound algorithm from results of incomplete execution of the dynamic algorithm. This modification will reduce branch and algorithms' complexity to $O\left(\left(\left|G\right|-k\right)!\right)$, where $k$ is the number of last iteration performed by the dynamic algorithm.
