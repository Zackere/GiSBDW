\documentclass[a4paper]{article}
\usepackage{subfiles}
\usepackage{graphicx}
\usepackage{amsmath}
\usepackage[export]{adjustbox}
\usepackage[margin=2.5cm]{geometry}
\usepackage{listings}
\usepackage{xcolor}
\usepackage{amsrefs}
\definecolor{codegreen}{rgb}{0,0.6,0}
\definecolor{codegray}{rgb}{0.5,0.5,0.5}
\definecolor{codepurple}{rgb}{0.58,0,0.82}
\definecolor{backcolour}{rgb}{0.95,0.95,0.92}

\lstdefinestyle{codestyle}{
	backgroundcolor=\color{backcolour},
	commentstyle=\color{codegreen},
	keywordstyle=\color{magenta},
	numberstyle=\tiny\color{codegray},
	stringstyle=\color{codepurple},
	basicstyle=\ttfamily\footnotesize,
	breakatwhitespace=false,
	breaklines=true,
	captionpos=b,
	keepspaces=true,
	numbersep=5pt,
	showspaces=false,
	showstringspaces=false,
	showtabs=false,
	tabsize=2,
	numbers=left
}
\lstset{style=codestyle}

\title{Calculating tree-depth\\Introduction}
\author{Dymitr Lubczyk \and Wojciech Replin \and Bartosz Różański}
\lstset{style=codestyle}

\begin{document}
\maketitle
\begin{abstract}
We present several methods for computing treedepth on both CPU and GPU. We also set goals for the future of our project. Presented algorithm is a combination of well known dynamic algorithm and our own innovatory approach using branch and bound technique.
\end{abstract}
\newpage
\tableofcontents
\newpage
\subfile{./introduction/main.tex}
\clearpage
\subfile{./typical_approaches/main.tex}
\clearpage
\subfile{./our_approach/main.tex}
\clearpage
\subfile{./expected_results/main.tex}
\clearpage
\subfile{./references/main.tex}
\end{document}
